\documentclass[]{article}

\usepackage{geometry} %um die dimensionen des Dokuments un deren Ränder anzupassen
	\geometry{
		a4paper,
		total={170mm,257mm},
		left=20mm,
		top=0mm,
	}

\title{Physik - Zusammenfassung fürs Abitur}
\date{2024}





\begin{document}
	\maketitle
	
	\begin{abstract}
		Abstract
	\end{abstract}
	
	\tableofcontents
	
	\section{Q1}
		\subsection{Gravitationsfeld}
		\subsection{Elektrisches Feld}
		\subsection{Magnetfeld}
		
	
	\section{Q2}
		\subsection{Induktion}
		\subsection{Wechselspannung}
		\subsection{Elektromagnetische Schwingungen}
	
	\section{Q3}
		\subsection{Elektronen in Feldern}
		\subsection{Photonen}
			\subsubsection{Teilchencharakter}
			\subsubsection{Wellencharakter}
	
	\section{Q4}
		\subsection{Röntgenstrahlung}
		\subsection{Kernphysik}
	
\end{document}