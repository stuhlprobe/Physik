\documentclass{article}
\usepackage{graphicx} % Required for inserting images

\title{Physik - Zusammenfassung fürs Abitur}
\author{Maximilian Penke, Emil Maihorn}
\date{January 2024}

\renewcommand*\contentsname{Gliederung}

\begin{document}

    \maketitle

    \begin{abstract}
        Dies ist eine Zusammenfassung für die Inhalte des Berliner Abiturs von 2024 im Fach Physik. Dabei ist des in die vier Halbjahre unterteilt, wobei es jeweils Differenzierungen gibt. Dafür werden die Inhalte der Einzelthemen erklärt, mit der allgemeinen Umsetzungsweise versehen und darauf folgend mit unterschiedlichen Beispielen veranschaulicht.
    \end{abstract}

    \tableofcontents

    \section{Allgemeine Hinweise zur Notation}

        \subsection{Koordinatensysteme}

        \subsection{Felder skizzieren}
        
        \subsection{Schaltkreise}

    \section{Q1: Felder}

        \subsection{Darstellung von Feldern}

        \subsection{Gravitationsfelder und Astronomie}

            \subsubsection{Keplersche Gesetze}

            \subsection{Gravitationskraft und Berechnung}

            \subsubsection{Gravitationsfeld}

        \subsection{Elektrische Felder und Kondensatoren}

    \section{Q2: Elektromagnetische Wellen}

    \section{Q3: Quantenphysik}

    \section{Q4: Kernphysik}
\end{document}




